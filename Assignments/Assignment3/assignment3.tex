\documentclass[12pt]{article}
\usepackage{fullpage,enumitem,amsmath,amssymb,graphicx,grffile,float,listings}



\begin{document}
    %Title Section
    \begin{flushleft}
    \LARGE CS229 Fall 2017\\
    \LARGE Problem Set \#3 Solutions:  Deep Learning \& Unsupervised Learning \\
    \textbf{\normalsize Author: LFhase \quad rimemosa@163.com}
    \end{flushleft} 
    \noindent
    \rule{\linewidth}{0.4pt}
    %Title Section

    %Problem and Solution
    \section*{A Simple Neural Network}
    \begin{enumerate}[label=(\alph*)]
        \item Using Chain Rule, we know that 
        $$\frac{\partial loss}{\partial w_{1,2}^{[1]}}
        = \frac{\partial loss}{\partial o}\frac{\partial o}{\partial h_2}\frac{\partial h_2}{\partial w_{1,2}^{[1]}}$$
        let $g(x)$ denote the sigmoid function, then we have
        $$g'(x) = g(x)(1-g(x))$$
        so $$\frac{\partial loss}{\partial w_{1,2}^{[1]}}
        =\frac{2}{m}\sum_{i=1}^m (o^{(i)}-y^{(i)})o^{(i)}(1-o^{(i)})w_2^{[2]}h_2^{(i)}(1-h_2^{(i)})x^{(i)}_1$$
        where 
        $$h_2^{(i)}=g(x^{(i)}_1w_{1,2}^{[1]}+x^{(i)}_2w_{2,2}^{[1]}+w_{0,2}^{[1]})$$
        \item let $(0.5,0.5)$, $(3.5,0.5)$, $(0.5,3.5)$ be the three poinst of the triangle.
        The forward transport in the neural network can be written in matrix form.
        $$
        \left[\begin{matrix}
            -1.5 & 3 & 0 \\
            -1.5 & 0 & 3 \\
            9 & -3 & 3\\
        \end{matrix}\right] 
        \times
        \left[\begin{matrix}
            1 \\
            x_1 \\
            x_2\\
        \end{matrix}\right]
        $$
        and
        $$
        \left[\begin{matrix}
            -1 & -1 & -1 & 2.33
        \end{matrix}\right] 
        \times
        \left[\begin{matrix}
            1 \\
            h_1 \\
            h_2\\
            h_3\\
        \end{matrix}\right]
        $$
        Once the point is in the triangle, the first product will be 
        $$
        \left[\begin{matrix}
            1 \\
            1 \\
            1 \\
        \end{matrix}\right]
        $$ 
        So the second product will be -0.67 and the final result will be 0.
        Otherwise, the second product will be larger or equal to 0.33 and the final result will be 1.
        \item Using $f(x)=x$ as hidden layer activation function, we can see the neural network as \textbf{a simple neural network without hidden layer},
        who only has the \textbf{convex boundary} and can't deal with the problem described in statement.
    \end{enumerate}

    \section*{EM for MAP estimation}
    The whole process is the same like what discussed in lecture notes. \\
    Firstly, we have log-likelihood:
    \begin{equation*}
        \begin{split}
            l(\theta) &= \sum_{i=1}^m log[\sum_{z^{(i)}}Q_i(z^{(i)})\frac{p(x^{(i)},z^{(i)}|\theta)}{Q_i(z^{(i)})}] + logp(\theta) \\
            &\geq \sum_{i=1}^m \sum_{z^{(i)}}Q_i(z^{(i)})log\frac{p(x^{(i)},z^{(i)}|\theta)}{Q_i(z^{(i)})} + logp(\theta)
        \end{split}
    \end{equation*}
    So if we set$$Q_i(z^{(i)})=p(z^{(i)}|x^{(i)},\theta)$$
    According to Jensen's Inequality, we have
    $$l(\theta)=\sum_{i=1}^m \sum_{z^{(i)}}Q_i(z^{(i)})log\frac{p(x^{(i)},z^{(i)}|\theta)}{Q_i(z^{(i)})} + logp(\theta)$$
    Then we get EM-step as below: \\
    E-step:$$Q_i(z^{(i)})=p(z^{(i)}|x^{(i)},\theta)$$
    M-step:$$\theta={argmax}_{\theta}[\sum_{i=1}^m \sum_{z^{(i)}}Q_i(z^{(i)})log\frac{p(x^{(i)},z^{(i)}|\theta)}{Q_i(z^{(i)})} + logp(\theta)]$$
    In our assumption, the M-step is tractable. Then we have
    \begin{equation*}
        \begin{split}
            l(\theta^{(t+1)}) 
            &\geq \sum_{i=1}^m \sum_{z^{(i)}}Q_i(z^{(i)})log\frac{p(x^{(i)},z^{(i)}|\theta^{(t+1)})}{Q_i(z^{(i)})} + logp(\theta^{(t+1)}) \\
            &\geq \sum_{i=1}^m \sum_{z^{(i)}}Q_i(z^{(i)})log\frac{p(x^{(i)},z^{(i)}|\theta^{(t)})}{Q_i(z^{(i)})} + logp(\theta^{(t)}) \\
            &= l(\theta^{(t)}) 
        \end{split}
    \end{equation*}
    The likelihood will increase monotonically with each iteration of the algorithm.

    \newpage
    \section*{EM application}
    \begin{enumerate}[label=(\alph*)]
        \item \begin{enumerate}[label=(\roman*)]
            \item Since we have $x^{(pr)} = y^{(pr)} + z^{(pr)} + \epsilon^{(pr)}$, then $X\sim N(\mu_p+\nu_r,\sigma^2+\sigma_p^2+\tau_r^2)$
            So the joint distribution have the mean vector and covariance matrix as below:
            $$
            \left[\begin{matrix}
                \mu_p \\
                \nu_r \\
                \mu_p+\nu_r \\
            \end{matrix}\right]
            $$ 
            and
            $$
            \left[\begin{matrix}
                \sigma_p^2  & 0 & \sigma_p^2 \\
                0 & \tau_r^2 & \tau_r^2 \\
                \sigma_p^2 & \tau_r^2 & \sigma^2+\sigma_p^2+\tau_r^2\\
            \end{matrix}\right]
            $$ 
            \item Using the formula in the notes, we have the mean vector and covariance matrix as below:
            $$
            \mu_Q = 
            \left[\begin{matrix}
                \mu_p \\
                \nu_r \\
            \end{matrix}\right]
            +
            \left[\begin{matrix}
                \sigma_p^2 \\
                \tau_r^2 \\
            \end{matrix}\right]\frac{x^{(pr)}-(\mu_p+\nu_r)}{\sigma^2+\sigma_p^2+\tau_r^2}
            $$ 
            and
            $$
            \Sigma_Q = 
            \left[\begin{matrix}
                \sigma_p^2  & 0  \\
                0 & \tau_r^2 \\
            \end{matrix}\right]
            -
            \left[\begin{matrix}
                \sigma_p^2 \\
                \tau_r^2 \\
            \end{matrix}\right]\frac{            
                \left[\begin{matrix}
                    \sigma_p^2 & \tau_r^2
            \end{matrix}\right]}{\sigma^2+\sigma_p^2+\tau_r^2}
            $$
            The expression is:
            $$Q_{pr}(y^{(pr)},z^{(pr)}) = \frac{1}{\sqrt{{2\pi}^2|\Sigma_Q|}}exp(-\frac{1}{2}
            (\left[\begin{matrix}
                y^{(pr)} \\
                z^{(pr)} \\
            \end{matrix}\right]-\mu_Q)^T 
            \Sigma_Q^{-1}
            (\left[\begin{matrix}
                y^{(pr)} \\
                z^{(pr)} \\
            \end{matrix}\right]-\mu_Q)
            )$$
        \end{enumerate}
        \item We want to maxmize the lower bound of the log-likelihood function:
        \begin{equation*}
            \begin{split}
                \Theta &= argmax_\Theta \sum_p \sum_r E_{(y^{(pr)},z^{(pr)})\sim Q_{pr}}[logp(x^{(pr)},y^{(pr)},z^{(pr)})] \\
                &= argmax_\Theta  \sum_p \sum_r E
                [
                    log\frac{1}{{2\pi}^{3/2}\sigma \sigma_p \tau_r}
                    -\frac{1}{2\sigma_p^2}(y^{(pr)}-\mu_p)^2
                    -\frac{1}{2\tau_r^2}(z^{(pr)}-\nu_r)^2\\
                    &-\frac{1}{2\sigma^2}(x^{(pr)}-y^{(pr)}-z^{(pr)})^2
                ]\\
                &=argmax_\Theta  \sum_p \sum_r E
                [
                    log\frac{1}{\sigma_p \tau_r}
                    -\frac{1}{2\sigma_p^2}(y^{(pr)}-\mu_p)^2
                    -\frac{1}{2\tau_r^2}(z^{(pr)}-\nu_r)^2
                ]
            \end{split}
        \end{equation*}
        Then we calculate the derivatives of each parameter and set them to zero to get the update value.
        $$\mu_p = \frac{1}{PR} \sum_p \sum_r {\mu_Q}_1$$
        $$\nu_r = \frac{1}{PR} \sum_p \sum_r {\mu_Q}_2$$
        $$\sigma_p^2 =\frac{1}{PR}\sum_p \sum_r ({\Sigma_Q}_{11}+{\mu_Q}_1^2-2{\mu_Q}_1\mu_p+\mu_p^2) $$
        $$\tau_r^2 =\frac{1}{PR}\sum_p \sum_r ({\Sigma_Q}_{22}+{\mu_Q}_2^2-2{\mu_Q}_2\nu_r+\nu_r^2) $$
    \end{enumerate}

    \section*{KL divergence and Maximum Likelihood}
    \begin{enumerate}[label=(\alph*)]
        \item  
        \begin{equation*}
            \begin{split}
                KL(P||Q) &= \sum_x P(x)-log\frac{Q(X)}{P(X)}\\
                &\geq  -log\sum_x(P(x)\frac{Q(X)}{P(X)})\\
                &= -log\sum_x Q(x)\\
            \end{split}
        \end{equation*}
        Since $\sum_x  Q(x) = 1$, so $KL(P||Q)\geq 0$. \\
        If $P=Q$, it's easily to see that $KL(P||Q)= 0$. \\
        Because $-log(x)$ is a strictly convex function, so we have the $=$ when $X=E[X]$ with probability 1 where $X=\frac{Q}{P}$.
        \item 
        \begin{equation*}
            \begin{split}
                KL(P(X)||Q(X)) + KL(P(Y|X)||Q(Y|X))&= \sum_x P(x) log\frac{P(x)}{Q(x)} + \sum_x P(x)(\sum_yP(y|x)log\frac{P(y|x)}{Q(y|x)})\\
                &=\sum_x P(x) (\sum_yP(y|x)log(\frac{P(x)}{Q(x)} \frac{P(y|x)}{Q(y|x)})\\
                &=\sum_x \sum_y P(x,y) log(\frac{P(y,x)}{Q(y,x)}) \\
                &=KL(P(Y,X)||Q(Y,X))
            \end{split}
        \end{equation*}
        \item 
        \begin{equation*}
            \begin{split}
                KL(\hat{P}||P_{\theta})&= \sum_x \hat{P} log\frac{\hat{P}(x)}{P_\theta (x)} \\
                &= -\sum_x \frac{1}{m} \Sigma_{i=1}^m \boldsymbol{1}\{x^{(i)}=x\} log\frac{P_\theta (x)}{\Sigma_{i=1}^m \boldsymbol{1}\{x^{(i)}=x\}}\\
                &= -\frac{1}{m} \Sigma_{i=1}^m logP_\theta (x^{(i)})
            \end{split}
        \end{equation*}
        So adjust $\theta$ to minmize the KL is equivalent to maxmize the log-likelihood function.
    \end{enumerate}
\end{document}